\documentclass[11pt,twoside,letterpaper]{article}
%NOTE: This report format is 

\newcommand{\reporttitle}{Reporte de entrenamiento...}
\newcommand{\reportauthorOne}{Nombre del Estudiante o estudiantes}
\newcommand{\cidOne}{your id number}
\newcommand{\reportauthorTwo}{Nombre del Asesor}
\newcommand{\cidTwo}{your id number}
\newcommand{\reporttype}{Coursework}
\bibliographystyle{plain}

% include files that load packages and define macros
\input{includes} % various packages needed for maths etc.
\input{notation} % short-hand notation and macros

%%%%%%%%%%%%%%%%%%%%%%%%%%%%

\begin{document}
% front page
% Last modification: 2016-09-29 (Marc Deisenroth)
% Modification for UW: 2017-05-22 (jphickey)
\begin{titlepage}

\newcommand{\HRule}{\rule{\linewidth}{0.5mm}} % Defines a new command for the horizontal lines, change thickness here


%----------------------------------------------------------------------------------------
%	LOGO SECTION
%----------------------------------------------------------------------------------------



\begin{center} % Center remainder of the page

%----------------------------------------------------------------------------------------
%	HEADING SECTIONS
%----------------------------------------------------------------------------------------

\includegraphics[width = 7cm]{./figures/logo-ipn}\\[1.5cm] 
\textbf{\Large Instituto Politécnico Nacional}\\[1.5cm] 
\textsc{\large Centro de Innovación y desarrollo tecnológico en cómputo}\\[0.95cm] 
%\textsc{\textsc{\Large Laboratorio de visión computacional}}\\[1.0cm] 

%----------------------------------------------------------------------------------------
%	TITLE SECTION
%----------------------------------------------------------------------------------------

\HRule \\[0.4cm]
{ \huge \bfseries \reporttitle}\\ % Title of your document
\HRule \\[1.5cm]
\end{center}
%----------------------------------------------------------------------------------------
%	AUTHOR SECTION
%----------------------------------------------------------------------------------------

%\begin{minipage}{0.4\hsize}
\begin{flushleft} \large
\textit{Authors:}\\
\reportauthorOne~(ID: \cidOne)\\ % Your name
\reportauthorTwo~(ID: \cidTwo)\\ % Your name
\end{flushleft}
\vspace{4cm}
\makeatletter
Date: \@date 

\vfill % Fill the rest of the page with whitespace



\makeatother


\end{titlepage}




%%%%%%%%%%%%%%%%%%%%%%%%%%% table of content
%If a table of content is needed, simply uncomment the following lines
%\tableofcontents
%\newpage

%%%%%%%%%%%%%%%%%%%%%%%%%%%% Main document
%\section*{Note:}
%\emph{Este documento es la plantilla para el reporte de los experimentos. No modificar el original. Se debe hacer una copia en su propio repositorio.}
\section{Resumen del experimento}

\subsection{Problemática general}

\emph{Proveer una descripción general del problema. Usar ecuaciones en caso de que exista un modelo matemático que lo defina. El problema debe estar acotado al experimento no a la tesis completa.}

Ejemplo: Clasificar imágenes de perros y gatos.

\subsection{Arquitectura de la red}

\emph{Describir la arquitectura de la red.}

\subsection{Preguntas de investigación}

\emph{Escribir la o las preguntas de investigación que se desean resolver. En caso de existir mas de una éstas se deben listar a continuación.}

%¿La red neuronal puede clasificar puentes?%Escribir la o las preguntas de investigación que se desean resolver

\begin{description}
\item [Pregunta 1:] ¿Cuáles son los hiperámetros que permiten alcanzar....?. 
\item [Pregunta 2:] Escribir la pregunta de investigación.
\end{description}

%\subsubsection*{Question 1:}
%¿El puente está completo?

%\subsubsection*{Question 2:  \label{Q2}}
%¿El puente tiene piezas faltantes?

\subsection{Hipótesis}

\emph{Poner la hipótesis como una relación de dos variables. Usualmente en aprendizaje profundo suponemos que cierta arquitectura o ciertos parámetros mejorarán el desempeño en una tarea determinada.}

Si utilizamos una búsqueda por rejilla de los hiperámetros podemos alcanzar un rendimiento de...

\section{Diseño de experimento (optimización de hiper-parámetros)}

\emph{En esta sección se describe la optimización de hyperparámetros.}

\subsection{Métricas de evaluación}

\emph{Describir las métricas que se van a usar para comparar la efectividad de cada tratamieto en cada sujeto.}

\begin{itemize}
    \item Exactitud (Accuracy): Mide .....
    \item Recuerdo (Recall): Mide... \\
    \vdots
    \item Métrica N: Mide ...
\end{itemize}

\subsection{Factores primarios}

\emph{Son los factores que al combinarse determinan el desempeño de un tratamiento.}

\begin{table}[h]
\centering
\begin{tabular}{|l|l|}
\hline
\textbf{Factor}     & \textbf{Valores posibles}   \\ \hline
Learning rate & $1 \times 10^{-3}$,  $1 \times 10^{-4}$    \\ \hline
Optimizador & Adam, SGD    \\ \hline
Tamaño de lote & 64, 32 \\ \hline
Semillas & A, B \\ \hline
\end{tabular}
\caption{Factores primarios}
\end{table}

\subsection{Factores ortogonales}

\emph{Los factores ortogonales son independientes de los otros factores definidos.}

Para disminuir la complejidad del experimento, consideraremos como factores ortogonales: batch size y número de épocas. Si bien tienen cierta influencia en el aprendizaje, ésta es menor que los demás parámetros. En un experimento aislado se determinan los mejores valores. 

\begin{table}[h]
\centering
\begin{tabular}{|l|l|}
\hline
\textbf{Factor}     & \textbf{Valor}   \\ \hline
Épocas     & 50    \\ \hline
Inicialización & Aleatoria \\ \hline
\end{tabular}
\caption{Factores ortogonales}
\end{table}

\subsection{Grid search u otra búsqueda}

\emph{Nota: En este ejemplo se pone grid search pero en caso de usar otro método de optimización se debe reemplazar esta sección por el otro método, e.g. optimización evolutiva.}

\emph{Primero se deben definir que factores se van a usar asi como los valores que tomará cada factor. Una vez determinados se hace la combinación completa en la tabla.}

\begin{table}[h]
\centering
\begin{tabular}{|l|l|l|l|}
\hline
ID                    & Factor 1 & $\dots$ & Factor N \\ \hline
1                     &          &                      &           \\ \hline
$\vdots$ &          &                      &               \\ \hline
\end{tabular}
\caption{Rejilla de búsqueda (Grid search).}
\end{table}

\section{Requerimientos del experimento}

\emph{Definir que requerimientos son necesarios para llevar a cabo el experimento. Al menos se debe especificar: requerimiento de procesamiento (GPU, CPU, NPU), requerimiento espacial (RAM), requerimiento temporal (tiempo de entrenamiento esperado para todo el experimento).}


\section{Análisis de resultados}

\emph{¿Se comprobó la hipótesis?}

\emph{¿Se realizó el experimento de acuerdo a los requerimientos?}

\emph{De no finalizar el experimento ¿Por qué no se finalizó?}



\newpage
\bibliography{mybib}


\end{document}
%%% Local Variables: 
%%% mode: latex
%%% TeX-master: t
%%% End: 
